\documentclass[12pt]{article}

\usepackage{sbc-template}
\usepackage{graphicx,url}

%\usepackage[brazil]{babel}   
\usepackage[utf8]{inputenc}  

     
\sloppy

\title{Maturidade no processo de desenvolvimento de software\\ Qualidade de código contínua }

\author{João Carlos  Brasileiro Stefenon de Almeida\inst{1}}


\address{
	Universidade do Vale do Rio dos Sinos
  (UNISINOS)\\
  São Leopoldo -- RS -- Brazil  
\email{jcbrasileiro@hotmail.com}
}

\begin{document} 

\maketitle

\begin{abstract}
  This meta-paper describes the style to be used in articles and short papers
  for SBC conferences. For papers in English, you should add just an abstract
  while for the papers in Portuguese, we also ask for an abstract in
  Portuguese (``resumo''). In both cases, abstracts should not have more than
  10 lines and must be in the first page of the paper.
\end{abstract}
     
\begin{resumo} 
  Este meta-artigo descreve o estilo a ser usado na confecção de artigos e
  resumos de artigos para publicação nos anais das conferências organizadas
  pela SBC. É solicitada a escrita de resumo e abstract apenas para os artigos
  escritos em português. Artigos em inglês deverão apresentar apenas abstract.
  Nos dois casos, o autor deve tomar cuidado para que o resumo (e o abstract)
  não ultrapassem 10 linhas cada, sendo que ambos devem estar na primeira
  página do artigo.
\end{resumo}



\part{INTRODUÇÃO}

 Uma das mudanças mais impactantes no processo de desenvolvimento de software das ultimas décadas, foi a concepção e absorção das metodologias ágeis de software, seja SCRUM, Extreme Programming (XP), Test Driven Development (TDD), Lean
Software Development, Kanban, etc, todas foram criadas a partir da filosofia AGILE \cite{MANISFESTOAGILE}, surgido a partir do esforço de várias pessoas que lidavam com o processo de software na década de 1990, com o objetivo de definir uma abordagem mais efetiva e eficiente para o desenvolvimento de software. Apesar dos conceitos já existirem a quase 20 anos, a importância e a implementação dessas ideias ainda são muito pouco exploradas no Brasil, segundo Coleman Parkes Research \cite{COLEMANPARKES_AGILE_2017} ; durante uma pesquisa; entrevistou cerca de 1.770 executivos de tecnologia da informação em 21 países, incluindo 76 brasileiros, mostra que somente 6\% das empresas têm utilizado fortemente, “transformando toda a organização para abranger os princípios de agilidade”.

 A partir dessa nova filosofia, a importância da qualidade no desenvolvimento criou novos pontos de vistas, aumentando significativamente também tanto na concepção, quanto no decorrer das outras fases, incluindo durante o desenvolvimento, e não mais apenas no final com um simples objetivo de execução. Um das mudanças foi a inclusão e a utilização do conceito de \textbf{MPV}, sigla de \textit{Minimum Viable Product}, que significa produto mínimo viável - conceito popularizado por Eric Ries \cite{ERICRIES_THELEAN} - que trouxe a importância de entregar e garantir de modo incremental, funcionalidades importantes para o cliente/negocio, ressaltando não apenas garantir um software útil, mas também visando a maleabilidade do negocio, as possibilidades de mudanças, seja de prioridades de funções, ou mesmo do próprio objetivo fim. A adoção desses pequenos entregáveis , gerou e exigiu; finalmente; uma maturidade no desenvolvimento, criando, por exemplo, a necessidade de artefatos que evidenciassem e garantissem desdo inicio dois grande aspectos: (1) Qualidade e (2) Garantia que cada entregável operasse corretamente.

 
 
 
 -
 
1º e 2º parágrafo: apresentação do tema dentro de um contexto.
-- Qualidade do código / garantia

3º parágrafo: delimitação do tema, apresentado através do problema de pesquisa do seu TCC.
-- Código ruim e "Ice cream" problem
 
4º parágrafo: apresente possíveis respostas para o problema de pesquisa levantado, ou seja, as hipóteses.
Aplicação das principios OO e CLEAN CODE.
Segregação das testes unitarios e testes de integração.

5º parágrafo: em poucas palavras, fale sobre o objetivo geral do trabalho e também dos específicos. Eles são ingredientes fundamentais para o trabalho.

6º parágrafo: apresente a relevância do seu trabalho acadêmico, identificando a importância dele para a sociedade ou comunidade científica. Isso é o que chamamos de justificativa.

7º parágrafo: descreva, em poucas palavras, qual metodologia foi utilizada. Foi pesquisa bibliográfica ou de campo? Você deve especificar o procedimento de forma concisa.

8º parágrafo: apresente a estrutura do trabalho, ou seja, como ele está dividido em capítulos. Lembre-se de falar, resumidamente, sobre o que se trata cada capitulo.




\part{REVISÃO BIBLIOGRÁFICA} \label{sec:revisaobibliografica}

\section{\textit{Clean Code}} \label{sec:cleancode}

Learning to write clean code is hard work. It requires more than just the knowledge of
principles and patterns. You must sweat over it. You must practice it yourself, and watch
yourself fail. You must watch others practice it and fail. You must see them stumble and
retrace their steps. You must see them agonize over decisions and see the price they pay for
making those decisions the wrong way.

\section{\textit{Object-Oriented Design Principles (OODP)}} \label{sec:oopd}

\subsection{\textit{General Responsibility Assignment Software Principles (GRASP)}} \label{sec:grasp}

\begin{itemize}
	\item Controller
	\item Creator
	\item Indirection
	\item Information Expert
	\item High Cohesion
	\item Low Coupling
	\item Polymorphism
	\item Protected Variations
	\item Pure Fabrication
\end{itemize}

\subsection{\textit{S.O.L.I.D principles}} \label{sec:solid}
\begin{itemize}
	\item Single responsibility principle
	\item Open closed principle
	\item Liskov substitution principle
	\item Interface segregation principle
	\item Dependency Inversion principle
\end{itemize}

\subsection{\textit{Law of Demeter principles}} \label{sec:law_of_demeter}
\subsection{\textit{Don't Repeat Yourself (DRY)}} \label{sec:law_of_demeter}
\subsection{\textit{Favor Composition over Inheritance.}} \label{sec:law_of_demeter}
\subsection{\textit{Keep it Simple, Stupid. (KISS)}} \label{sec:law_of_demeter}



\section{\textsl{Test-driven development (TDD)}} \label{sec:tdd}
Test-driven development (TDD) is a software development process that relies on the repetition of a very short development cycle: first the developer writes an (initially failing) automated test case that defines a desired improvement or new function, then produces the minimum amount of code to pass that test, and finally refactors the new code to acceptable standards.

The process can be defined as such:

    Write a failing unit test
    Make the unit test pass
    Refactor

Repeat this process for every feature, as is necessary.

\part{Trabalhos Relacionados} \label{sec:trabalhosrelacionados}

\bibliographystyle{sbc}
\bibliography{TCC-UNISINOS-ADS}

\end{document}
